\wt{122} Every Cauchy sequence in $\R$ converges.
\wde{123} $\lim \inf x_n = \lim_{n \to \infty} l_n = \lim_{n \to \infty} \inf_{k \geq n} x_k$.
$\lim \sup x_n = \lim_{n \to \infty} S_n = \lim_{n \to \infty} \sup_{k \geq n} x_k$.
\we{124} Lim inf is the smallest subsequential limit of $(x_n)$.
Lim sup is the largest subsequential limit of $(x_n)$.
$(x_n)$ converges iff $\lim \inf x_n = \lim \sup x_n$.
\wde{125} Let $X = \R$ and $d$ be the standard metric.
A subset $K$ of $\R$ is said to be compact iff every sequence of elements of $K$
has a subsequence that converges to an element of $K$.
\we{126}
Compact: $[a,b], \R \cup \{-\infty, \infty\}, \emptyset$, not: $(0,1), \R$.
\wt{Including Heine-Borel} ORS
(1) (Heine-Borel): compact iff closed and bounded (also on $\R^n$ with Euclidean metric).
(2) compact iff every open cover has a finite subcover.
\we{128} In $\R^n$ with the Euclidean metric, compact iff closed and bounded.
\wt{} ORS $K$ compact then $f: K \to \R$ cont. then $f$ is bounded and attains its bounds.
$[a,b]$ compact.