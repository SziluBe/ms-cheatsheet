\wt{185} ORS every open set can be written as a countable union of disjoint open intervals.
Let $G \subseteq \R$ open,
def. eq. rel.: $x \sim y$ iff $\exists$ open interval $I \subseteq G$ s.t. $x,y \in I$.
G is the union of the equivalence classes of $\sim$.
Fix $x \in G$, let $m = \inf \{a \in \R : [a] \subseteq G\}$,
$M = \sup \{b \in \R : [b] \subseteq G\}$.
Then
$[x] \subseteq (m,M)$,
$(m,M) \subseteq G$.
\we{186} $[x]$ is the largest open interval that contains $x$ and is contained in $G$.
\wt{187} $(X, d_X), (Y, d_Y)$ metric spaces, $D$ a dense subset of $X$,
$f, g: X \to Y$ cont. s.t. $f(x) = g(x)$ for all $x \in D$.
Then $f = g$.
\we{188} $f: \R \to \R$ cont., $f(x) = x^2$ for all $x \in \Q$.
Then $f(x) = x^2$ for all $x \in \R$.
\wt{189} $(X, d_X), (Y, d_Y)$ metric spaces, $D \subseteq X$ dense,
$f: D \to Y$ UC, $Y$ complete.
Then $f$ has a unique cont. (and UC) extension to $X$.
\wt{190} $(X,d)$ a metric space, $F$ a non-emmpty subset of $X$ and $d_F$ the induced metric on $F$.
If the MS $(F, d_F)$ is complete, then $F$ is closed in $X$.
The converse is false.
\wt{192} $(X,d)$ a complete MS, $\emptyset \neq F \subseteq X$ and $d_F$ the induced metric on $F$.
If $F$ closed, then $(F, d_F)$ is complete.
\wt{193} $(X,d)$ a complete metric space, $\emptyset \neq A \subseteq X$. Then
(1) $(\overline{A}, d_A)$ is complete.
(2) If $A \subseteq B \subseteq X$ and $(B, d_B)$ is complete, then $\overline{A} \subseteq B$.
